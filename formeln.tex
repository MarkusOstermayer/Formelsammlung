\documentclass[10pt,a4paper]{article}
\usepackage[utf8]{inputenc}
\usepackage[ngerman]{babel}
\usepackage[T1]{fontenc}
\usepackage{amsmath}
\usepackage{amsfonts}
\usepackage{amssymb}
\usepackage{graphicx}
\usepackage{lmodern}
\usepackage{physics}
\usepackage[left=1cm,right=1cm,top=2cm,bottom=1.5cm]{geometry}
\usepackage{siunitx}
\usepackage{fancyhdr}
\usepackage{enumerate}
\usepackage{mhchem}
\usepackage{mathtools}
\usepackage{graphicx}
\usepackage{float}
\usepackage{xcolor}
\usepackage{mdframed}
\usepackage{csquotes}
\usepackage{trfsigns}
\usepackage{capt-of}
\usepackage{comment}

\sisetup{locale=DE}
\sisetup{per-mode = symbol-or-fraction}
\sisetup{separate-uncertainty=true}
\DeclareSIUnit\year{a}
\DeclareSIUnit\clight{c}
\mdfdefinestyle{exercise}{
	backgroundcolor=black!10,roundcorner=8pt,hidealllines=true,nobreak
}

\begin{document}
\twocolumn
\pagestyle{fancy}
\lhead{Physik für Elektrotechniker \\ Formelsammlung}

	\section{Einführung}
	\subsection{Fehlerfortpflanzung}
		\begin{mdframed}[style=exercise]
			\begin{math}
				\Delta f = \abs{\frac{\partial f(x,y,z)}{\partial x}} * \abs{\Delta x} + \abs{\frac{\partial f(x,y,z)}{\partial y}} * \abs{\Delta y} + \abs{\frac{\partial f(x,y,z)}{\partial z}} * \abs{\Delta z}
			\end{math}
		\end{mdframed}

	\section{Mechanik}

	\subsection{Kinematik}
	\begin{mdframed}[style=exercise]
		\begin{math}
			a_{1D} = \frac{\partial v}{\partial t} = \dot v; a = \frac{\partial}{\partial t}(\frac{\partial s}{\partial t}) = \ddot s
			\newline
			v_{1D} = \frac{\partial s}{\partial t} = \dot s
			\newline
			\vec v_{3D} = \frac{\partial\vec r}{\partial t} = \vec\dot r = \begin{pmatrix} \dot x \\ \dot y \\ \dot z \end{pmatrix}
			\newline
			\vec a_{3D} = \vec\dot v = \ddot\vec r = \begin{pmatrix} \ddot x \\ \ddot y \\ \ddot z \end{pmatrix}
			\newline
			\abs{\vec a_{zp}} = \frac{v^2}{r} = \omega^2 * r
			\newline
			\omega = \frac{2*\pi}{f}
		\end{math}
	\end{mdframed}

	\subsection{Energie}
	\begin{mdframed}[style=exercise]
		\begin{math}
			\text{Energieunterschied:} \Delta E = E_{nachher} - E_{vorher}
			\newline
			\text{potentielle Energie:} E_{pot} = m*g*h
			\newline
			\text{kinetische Energie:} E_{kin} = \frac{m*v^2}{2}
			\newline
			\text{Elektrische Energie:} E_{el} = \frac{1}{2} * k * s^2
			\newline
			\text{Rotationsenergie:} E_{rot} = \frac{1}{2} * J * \omega^2
			\newline
			\text{Energie des Kondesators:} E_{Kondensator} = \frac{Q}{2*C} = \frac{C*U^2}{2}
			\newline
			E_{Spule} = \frac{L*I^2}{2}
			\newline
			\text{Protonenenergie:} E_{Photon} = h * f
		\end{math}
	\end{mdframed}

	\subsection{Kräfte, Impulse und Arbeit}
	\begin{mdframed}[style=exercise]
		\begin{math}
			\text{zweites newton'sches Axiom:} F = m * a
			\newline
			\text{drittes newton'sches Axiom:} F_{12} = - F_{21}
			\newline
			\text{Superposition von Kräften:} \sum_{i} F_{i}
			\newline
			\text{Gewichtskraft:} \vec F_{G} = m * \vec g
			\newline
			\text{Federkraft:} F = -k * s
			\newline
			\text{Zentripedalkraft:} F_{zp} = -m * \omega^2 * r
			\newline
			\text{Haftreibung:} F_{R} = \mu * F_{N}
			\newline
			\text{Flüssigkeitsreibung:} F_{R} = \chi * \vec v
			\newline
			\text{Impuls:} \vec p = m * \vec v
			\newline
			\text{Kraftstoß:} \int\limits_{t_{1}}^{t_{2}} F_{t} \mathrm{d}t = \int\limits_{p_{1}}^{p_{2}} \mathrm{d}t = \vec p_{2} - \vec p_{1} = \vec\Delta p
			\newline
			\text{Beschleunigungsarbeit:} W_{12} = \frac{m * (v_{2}^2 - v_{1}^2)}{2}
			\newline
			\text{Hubarbeit gegen die Gravitation:} F = -G * \frac{m*M}{r^2}\vec e_{r}
			\newline
			\Rightarrow W_{12} = G*m*M*(\frac{1}{r_{1}} - \frac{1}{r_{2}})
		\end{math}
	\end{mdframed}

	\subsection{Leistung}
	\begin{mdframed}[style=exercise]
		\begin{math}
			\text{Leistung:} P = \frac{\Delta W}{\Delta t} bzw. P = \frac{\partial W}{\partial t} = \vec F * \vec v
		\end{math}
	\end{mdframed}

	\subsection{Stöße}
	\begin{mdframed}[style=exercise]
		\begin{math}
			\text{Impuls}: m_{1}*v_{1} + m_{2}*v_{2} = m_{1}*v_{1}\prime + m_{2}*v_{2}\prime
			\newline
			Energie: \frac{m_{1} * v_{1}^2}{2} + \frac{m_{2} * v_{2}^2}{2} = \frac{m_{1} * v_{1}^2\prime}{2} + \frac{m_{2} * v_{2}^2\prime}{2} + \Delta W
			\newline
			\text{Bei elastischen Stössen gilt:} \Delta W = 0
		\end{math}
	\end{mdframed}

	\subsection{Impulserhaltungssatz}
	\subsection{Arbeit für konstante Kraft}
	\subsection{Beschleunigungsarbeit}
	\subsection{Hubarbeit}
	\subsection{Leistung}
	\subsection{Energie}
	\subsection{kinetische Energie}
	\subsection{potentielle Energie}
	\subsection{Gerader Stoß}
	\subsubsection{elastisch}
	\subsubsection{unelastisch}
	\subsection{Drehmoment}
	\subsection{Drehimpulss}





	\section{Thermodynamik}

	\section{Elektrizität & Magnetismus}

	\section{Schwingungen & Wellen}

	\section{Optik}

	\section{Atomphysik}

	\section{Festkörperphysik}

	\begin{comment}

\section{Lineare zeitinvariante Systeme}
  \subsection{Eigenschaften}
  Eigenschaften LTI-Systeme
  \begin{mdframed}[style=exercise]
    \begin{enumerate}
      \item Stabilität\\
      $\abs{x(t)} < M < \infty \Rightarrow \abs{y(t)} < N < \infty$
      \item Linearität\\
      $W\qty{\sum_{k=1}^{N}a_n x_n(t)}=\sum_{n=1}^{N}W\qty{a_n x_n(t)}$
      \item Zeitinvarianz\\
      $W\qty{x(t-t_0)} =y(t-t_0)$
      \item Kausalität\\
      $t < 0 \Rightarrow x(t)=0 \land y(t)=0$
    \end{enumerate}
  \end{mdframed}

  \subsection{Systemantwort}
  Die Sprung-/Impulsantwort beschreibt Systemantwort vollständig
  \begin{mdframed}[style=exercise]
    \begin{align}
      y(t) &= \int^{\infty}_{-\infty} a(t-\tau) x'(\tau) \dd{\tau}\\
      a(t-\tau) &= W\qty{s(t-\tau)}\nonumber
    \end{align}
  \end{mdframed}
  \begin{mdframed}[style=exercise]
    \begin{align}
      y(t) &= \int_{-\infty}^{\infty} h(t-\tau) x(\tau) \dd{\tau}\\
      h(t-\tau) &= W\qty{\delta(t-\tau)}\nonumber
    \end{align}
  \end{mdframed}
  \subsection{Abtasttheorem}
  Durch die Abtastung wird das Spektrum von $f(t)$ unendlich oft um die Frequenzen $n\cdot \omega_a$ reproduziert.
  \begin{mdframed}[style=exercise]
    \begin{align}
      F_A(\omega) &= \frac{1}{T_A} \sum_{n=-\infty}^{\infty} F(\omega-n\omega_A)\\
      2\omega_g &\leq \omega_A\nonumber
    \end{align}
  \end{mdframed}
  \section{Transformationen}
  \subsection{Fourierreihe}
  \begin{mdframed}[style=exercise]
    \begin{align}
      f(t) &= \sum_{n=0}^{\infty} \qty[a_n \cos(n \omega_0 t) + b_n \sin(n \omega_0 t)]\\
      a_n &= \frac{2}{T}\int_{-T/2}^{T/2}f(t)\cos(n\omega_0 t)\dd{t}\nonumber\\
      b_n &= \frac{2}{T}\int_{-T/2}^{T/2}f(t)\sin(n\omega_0 t)\dd{t}\nonumber
    \end{align}
  \end{mdframed}
  \pagebreak
  \subsection{Fourierreihe, komplex}
  \begin{mdframed}[style=exercise]
    \begin{align}
      f(t) &= \sum_{n=-\infty}^{\infty} c_n e^{jn\omega_0 t}\\
      c_n &= \frac{1}{T} \int_{-T/2}^{T/2} f(t) e^{-jn\omega_0 t}\dd{t}\nonumber
    \end{align}
  \end{mdframed}
  \subsection{Fourierintegral}
  \begin{mdframed}[style=exercise]
    \begin{align}
      f(t) &= \frac{1}{2\pi} \int_{-\infty}^{\infty} F(\omega) e^{j\omega t} \dd{\omega}\\
      F(\omega) &= \int_{-\infty}^{\infty} f(t) e^{-j\omega t} \dd{t}
    \end{align}
  \end{mdframed}
  \subsubsection{Eigenschaften}
  \begin{mdframed}[style=exercise]
    \begin{enumerate}
      \item Linearität\\
      $a f_1(t) + b f_2(t) \laplace a F_1(\omega) + b F_2(\omega)$
      \item Zeitverschiebung\\
      $f(t-t_0) \laplace F(\omega)e^{-j\omega t_0}$
      \item Frequenzverschiebung\\
      $f(t) e^{\pm j\omega_0 t}\laplace F(\omega\mp \omega_0)$
      \item Faltung\\
      $f_1(t)*f_2(t)\laplace F_1(\omega)\cdot F_2(\omega)$\\
      $f_1(\omega)\cdot f_2(\omega)\laplace \frac{1}{2\pi}F_1(t)*F_2(t)$
    \end{enumerate}
  \end{mdframed}
  \subsection{DFT}
  \begin{mdframed}[style=exercise]
    \begin{align}
      x_n &= \frac{1}{N} \sum_{k=0}^{N-1} X_k\cdot e^{i 2 \pi k n / N}\\
      X_k &= \sum_{n=0}^{N-1} x_n\cdot e^{-i 2 \pi k n / N}
    \end{align}
  \end{mdframed}
  \subsubsection{FFT}
  \begin{center}
    \includegraphics[width=.35\textwidth]{butterfly}
    \captionof{figure}{FFT}
  \end{center}
  \pagebreak
  \subsection{Hilbert Transformation}
  \begin{mdframed}[style=exercise]
    \begin{align}
      x_{\mathrm{ht}}(t) &= x_{\mathrm{r}}(t) * h(t)\\
      H(\omega) &= -j \, \text{sgn}(\omega)
    \end{align}
  \end{mdframed}
  \subsection{z Transformation}
  \begin{mdframed}[style=exercise]
    \begin{align}
      X(z) &= \sum_{n=-\infty}^{\infty} x(n) z^{-n}\\
      x(n) &= \frac{1}{2\pi j} \oint_c X(Z) z^{n-1}\dd{z}
    \end{align}
  \end{mdframed}
  \subsubsection{Übertragungsfunktion}
  \begin{mdframed}[style=exercise]
    \begin{align}
      H(Z) &=\frac{Y(Z)}{X(z)}=\frac{\sum_{k=0}^q b_k z^{-k}}{\sum_{k=0}^p a_k z^{-k}}=k\frac{\prod_{k=1}^q (1-z_k z^{-1})}{\prod_{k=1}^p (1-p_k z^{-1})}
    \end{align}
  \end{mdframed}
  \subsubsection{Verschiebung im Zeitbereich}
  \begin{mdframed}[style=exercise]
    \begin{align}
      Y(z) &= \sum_{n=0}^{\infty} \qty[x(n-m)] z^{-n} =  z^{-m} X(z)\\
      Y(z) &= \sum_{n=0}^{\infty}\qty[x(n+m)]z^{-n} = z^{m}\qty[x(t)-\sum_{n=0}^{m-1} x(n) z^{-n}]
    \end{align}
  \end{mdframed}
  \section{Filter}
  \subsection{FIR}
  \begin{mdframed}[style=exercise]
    \begin{align}
      y[n] &= \sum_{k=0}^{q} b_k x(n-k)
    \end{align}
  \end{mdframed}
  \begin{center}
    \includegraphics[width=.4\textwidth]{fir}
  \end{center}
  \subsection{IIR}
  \begin{mdframed}[style=exercise]
    \begin{align}
      y[n] &= \sum_{k=0}^{q} b_k x(n-k) - \sum_{k=1}^{p} a_k y(n-k)
    \end{align}
  \end{mdframed}
  \begin{center}
    \includegraphics[width=.35\textwidth]{df1}
    \captionof{figure}{Direkt Form 1}
  \end{center}
  \begin{center}
    \includegraphics[width=.35\textwidth]{df2}
    \captionof{figure}{Direkt Form 2}
  \end{center}
  \section{Entropie}
  \subsection{Informationsgehalt}
  \begin{mdframed}[style=exercise]
    \begin{align}
      I_i &= \log_2 \frac{1}{p_i}
    \end{align}
  \end{mdframed}
  \subsection{Entropie}
  Mittlerer Informaionsgehalt
  \begin{mdframed}[style=exercise]
    \begin{align}
      H(s) &= E\qty{I} = -\sum_{i=0}^{n-1} p_i \log_2 p_i\\
      0 &\leq H(s) \leq \log_2 n
    \end{align}
  \end{mdframed}
  Für $H(s)=\log_2 n$ Gleichverteilung und völlige Ungewissheit.
\end{comment}
\end{document}
